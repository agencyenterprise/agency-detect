\documentclass[10pt,conference]{IEEEtran}
\usepackage{amsmath,amssymb}
\usepackage{graphicx}
\usepackage{cite}
\usepackage{hyperref}

\title{Bitwise Intelligence:\\A Blanket–Information Measure of Competence
\thanks{\hspace{1em}Builds on UAD \cite{uad2025} and ABC \cite{abc2025} frameworks for agent discovery and multi-agent cooperation analysis.}}

\author{Gunnar Zarncke\\
AE Studio}
\date{July 2025}

\begin{document}
\maketitle

%------------------------------------------------------------------
\begin{abstract}
We introduce \emph{B-IQ}, a task-agnostic competence metric grounded in the information flows across an agent's Markov blanket.  
B-IQ combines predictive and control mutual information and discounts memory cost and residual surprise.  
We derive physical bounds, show how B-IQ aggregates hierarchically, and provide empirical anchors spanning ants, human societies, and large-language-model (LLM) services.  
Efficiency (\(\eta\))—the rate at which additional B-IQ converts into relative growth—varies by 15 orders of magnitude across anchors.  
A tentative U-shaped scaling with system size appears, though the pattern may reflect sampling bias or hidden confounds.
\end{abstract}

%------------------------------------------------------------------
\section{Blanket Preliminaries (UAD)}
For any agent \(X\) discovered via Unsupervised Agent Discovery (UAD) \cite{uad2025}, variables factor as Markov blankets:
\(
I^X_t\!\to\!A^X_t\!\to\!S^X_{t+1},
\;
S_t^X\!\to\!I^X_{t+1},
\)
with externals \(E^X_t\). The UAD framework identifies coherent agents by locating boundaries where \(I(I^X_{t+1};E^X_{t+1}\mid S^X_t,A^X_t) \approx 0\), ensuring conditional independence across the Markov blanket.

We define the core information-theoretic quantities:
\begin{align}
I_{\mathrm{pred}} &= I(I_t;S_{t+1}), &
I_{\mathrm{ctrl}} &= I(A_t;E_{t+1}),\nonumber\\
S &= \mathbb E[-\!\log P(S_{t+1}\!\mid I_t)],\qquad
H(I_t)&=\text{entropy of }I_t.
\end{align}

%------------------------------------------------------------------
\section{Definition of B-IQ}
\label{sec:def}
Per time step (nat\,s\(^{-1}\)):
\begin{equation}
\boxed{
\mathrm{B\!-\!IQ}
= I_{\mathrm{pred}}
+ \alpha\,I_{\mathrm{ctrl}}
- \beta\,H(I_t)
- \gamma\,S.
}
\end{equation}

\subsection{Coefficient Choices \texorpdfstring{$\alpha,\beta,\gamma$}{(alpha,beta,gamma)}}
\label{sec:coeff}
The competence functional (Sec.~\ref{sec:def})
\[
\mathrm{B\!-\!IQ}
 = I_{\mathrm{pred}}
 + \alpha\,I_{\mathrm{ctrl}}
 - \beta\,H(I_t)
 - \gamma\,S
\]
contains three dimensionless weights; we adopt the following canonical values.

\begin{enumerate}
\item \textbf{Control weight $\alpha=1$.}  
Adding one nat to the causal channel $I_{\mathrm{ctrl}}$ reduces expected variational free‐energy by exactly one nat when the actuator is noise-free.  Setting $\alpha=1$ therefore places prediction and control on the same natural information scale.

\item \textbf{Surprise penalty $\gamma=1$.}  
Residual surprise $S=-\log P(S_{t+1}\mid I_t)$ already measures free‐energy (in nats).  A unit penalty $\gamma=1$ is thus the least-assumption choice.

\item \textbf{Memory cost $\beta\!\ll\!1$.}  
Sustaining one stored bit for one second costs at minimum $kT\ln2\approx3\times10^{-21}$ J (Landauer, 300 K).  A single cortical spike dissipates $\sim2\times10^{-13}$ J, eight orders higher.  We therefore assign a small constant $\beta\!\approx\!10^{-3}$ (brains) or $10^{-4}$ (modern DRAM), reflecting the marginal energetic burden of memory relative to functional gain.
\end{enumerate}

These choices are physically motivated yet remain tunable; Sec.~\ref{sec:sens} quantifies downstream sensitivity.

For this study we use \(\alpha=\gamma=1,\;\beta\!\ll1\) (memory cost minor for biological brains).

%------------------------------------------------------------------
\section{Physical Envelope}
\label{sec:bounds}
With input channel capacity \(C_{\mathrm{sens}}\) and actuator capacity \(C_{\mathrm{act}}\):
\begin{equation}
\begin{aligned}
0 &\le I_{\mathrm{pred}}\le \min\!\bigl\{C_{\mathrm{sens}},H(S_{t+1})\bigr\},\\
0 &\le I_{\mathrm{ctrl}}\le C_{\mathrm{act}},\\
0 &\le H(I_t)\le C_{\mathrm{mem}}^{\mathrm{rt}}
      \le \frac{P}{kT\ln2},\\
H(S_{t+1})-C_{\mathrm{sens}} &\le S\le H(S_{t+1}).
\end{aligned}
\end{equation}
Substituting gives\footnote{Derivation mirrors ABC Eq.\,(6); see also \cite{Tishby2000IB} for predictive information bounds.}
\begin{align}
-\beta C_{\mathrm{mem}}^{\mathrm{rt}}-\gamma H(S_{t+1})
\le \mathrm{B\!-\!IQ}
\le C_{\mathrm{sens}}+\alpha C_{\mathrm{act}}.
\end{align}

%------------------------------------------------------------------
\section{Efficiency $\eta$}
Following ABC, for two agents \(A,H\) competing for share \(w\):
\begin{equation}
\frac{\dot w_A}{w_A}-\frac{\dot w_H}{w_H}
=\eta\,\bigl(\mathrm{B\!-\!IQ}_A-\mathrm{B\!-\!IQ}_H\bigr),
\end{equation}
so
\(
w_A/w_H
\!=\!(w_A/w_H)_0 \exp[\eta\,\Delta B\,t].
\)
Empirical \(\eta\) is extracted as \(g/\Delta B\) where \(g\) is observed growth differential.

%------------------------------------------------------------------
\section{Multi-Agent Cooperativity Index}
\label{sec:cooperativity}

Building on UAD and attractor basin theory \cite{uad2025, abc2025}, we characterize when agents benefit from transparency versus opacity in their information disclosure. 
For agents \(i\) and \(j\), cooperation emerges when the \emph{cooperativity index} exceeds unity:

\begin{equation}
\kappa_{ij} = \frac{b_{ij} \cdot p_{ij} \cdot \rho_{ij}}{c_{ij}} > 1,
\end{equation}

where:
\begin{itemize}
\item $b_{ij}$: marginal benefit to agent $i$ per bit of information disclosed by agent $j$
\item $p_{ij}$: probability that $j$'s cooperative action reaches $i$'s sensory channel
\item $\rho_{ij}$: strategic correlation between agents (normalized mutual information $I(M^j;A^j)/H(A^j)$)
\item $c_{ij}$: marginal cost to agent $j$ per bit of cooperative disclosure
\end{itemize}

This generalizes Hamilton's rule to information-theoretic settings where \emph{relatedness} is measured by predictive correlation rather than genetic similarity.

\subsection{Transparency Weight}
Each agent $i$ assigns a \emph{transparency weight} $\gamma_{ij}$ to neighbor $j$:
\begin{equation}
\gamma_{ij} = \frac{\partial}{\partial I(M^j;A^j)} \left[-\mathbb{E}_q[-\log P(S^i \mid I^i)] - \lambda H(I^i)\right],
\end{equation}
positive for cooperation ($\kappa_{ij} > 1$), negative for strategic opacity ($\kappa_{ij} < 1$).

\subsection{Percolation of Cooperation}
In networks with degree distribution $P(d)$ and bond probability $\phi = \Pr(\kappa_{ij} > 1)$, a giant cooperative component emerges when:
\begin{equation}
\phi > \phi_c = \frac{\langle d \rangle}{\langle d^2 \rangle - \langle d \rangle}.
\end{equation}

This \emph{percolation threshold} determines whether cooperation spreads globally or remains fragmented in small clusters.

%------------------------------------------------------------------
\section{Empirical Anchors}
\label{sec:anchors}
Table \ref{tab:anchors} summarises initial anchors with \(\Delta B\) estimated from published IO rates and \(g\) from longitudinal studies.

\begin{table}[!ht]
\centering\footnotesize
\caption{Representative B-IQ anchors (nat\,s\(^{-1}\))}
\label{tab:anchors}
\begin{tabular}{lccc}
\hline
System & $\Delta B$ & $g$ (s$^{-1}$) & $\eta$ \\
\hline
Ant species & $1\times10^{4}$ & $7\!\times\!10^{-8}$ & $7\!\times\!10^{-12}$\\
Fire-ant colony & $7\!\times\!10^{7}$ & $5.7\!\times\!10^{-7}$ & $8\!\times\!10^{-15}$\\
GitHub Copilot & $5\!\times\!10^{2}$ & $2.3\!\times\!10^{-4}$ & $4.6\!\times\!10^{-7}$\\
Kiva warehouse & $1\!\times\!10^{9}$ & $2.2\!\times\!10^{-8}$ & $2\!\times\!10^{-17}$\\
Large US firms & $2\!\times\!10^{12}$ & $9.5\!\times\!10^{-10}$ & $5\!\times\!10^{-22}$\\
\hline
\end{tabular}
\end{table}

\subsection{Hierarchical Scaling Anchors}
\label{sec:hier_anchors}

Beyond single-agent measurements, we examine hierarchical systems across two domains: AI architectures and human organizations. Table \ref{tab:hierarchical} presents efficiency measurements across system scales.

\begin{table}[!ht]
\centering\footnotesize
\caption{Hierarchical B-IQ scaling anchors}
\label{tab:hierarchical}
\begin{tabular}{lccc}
\hline
System & Scale ($N$) & $\Delta B$ & $\eta$ \\
\hline
\multicolumn{4}{c}{\textbf{AI Hierarchies}} \\
\hline
Microservice guild & $10^1$--$10^2$ & $\sim10^3$ & $10^{-9}$ \\
Kubernetes cluster & $10^3$--$10^4$ & $\sim10^6$ & $10^{-14}$ \\
Multi-tenant platform & $10^5$--$10^6$ & $\sim10^9$ & $10^{-16}$ \\
Global AI society & $10^7$--$10^8$ & $\sim10^{12}$ & $10^{-15}$ \\
\hline
\multicolumn{4}{c}{\textbf{Human Hierarchies}} \\
\hline
Individual & $1$ & $3\times10^5$ & -- \\
Bands/tribes & $10^1$--$10^2$ & $\sim10^6$ & $10^{-11}$ \\
Chiefdoms & $10^3$--$10^4$ & $\sim10^8$ & $10^{-17}$ \\
Nation-states & $10^5$--$10^8$ & $\sim10^{11}$ & $10^{-14}$ \\
\hline
\end{tabular}
\end{table}

*Hierarchical Trend*: Both AI and human systems exhibit U-shaped efficiency curves, with peak efficiency at small scales ($N<10^2$), minimum efficiency at intermediate scales ($10^3$--$10^4$), and partial recovery at large scales ($N>10^5$) due to institutional coordination.

\subsection{System-Specific Derivations}
\label{sec:derivations}

\subsubsection{Ant Species}
\textbf{B-IQ Estimate}: $\Delta B \approx 10^4$ nat/s. Seidl \& Wehner (2008) demonstrate ants integrate visual, proprioceptive, and efference copy signals for navigation. Dauzere-Peres \& Wystrach (2024) show ants process multiple sensory modalities simultaneously for robust path planning. While these papers don't specify bit rates, we estimate individual ant sensory bandwidth $\sim10^3$ bit/s based on processing visual landmarks, chemical gradients, and tactile cues at $\sim$10 Hz with $\sim$10 bits/channel across $\sim$10 channels. Motor output $\sim10^2$ bit/s for movement direction, speed, and pheromone release patterns. Colony coordination \cite{gordon2020} adds $\sim10^1$ nat/s per individual through collective information processing. For $\sim10^3$ individuals: total B-IQ $\approx 10^4$ nat/s.

\textbf{Efficiency}: Growth rate $g \approx 7 \times 10^{-8}$ s$^{-1}$ from reproductive success data over seasonal cycles \cite{ant_nest_rate}. Efficiency $\eta = g/\Delta B \approx 7 \times 10^{-12}$.

\subsubsection{Fire-Ant Colonies}
\textbf{B-IQ Estimate}: $\Delta B \approx 7 \times 10^7$ nat/s. Tschinkel \cite{tschinkel2011}. documents fire ant territories ranging from small colonies to extensive foraging networks with complex underground tunnel systems extending >15m from nests. The study shows territory area positively correlates with colony size and forager population, but doesn't provide specific colony sizes. We estimate large colonies at $N \sim 10^5$ based on the described territorial scope and foraging complexity. Individual ant B-IQ $\sim 10^2$ nat/s, but coordination overhead reduces effective per-capita contribution. Non-linear scaling due to communication bottlenecks and foraging territory constraints: total B-IQ $\propto N^{0.7} \approx 7 \times 10^7$ nat/s.

\textbf{Efficiency}: Tschinkel \cite{tschinkel2011}. describes seasonal variations in colony growth with spring decline due to sexual production and fall recovery through worker production, but provides no quantitative growth rates. We estimate $g \approx 5.7 \times 10^{-7}$ s$^{-1}$ assuming annual territorial expansion cycles of $\sim$20\% based on the described seasonal dynamics: $g \approx \ln(1.2)/(365 \times 24 \times 3600) \approx 5.7 \times 10^{-7}$ s$^{-1}$. Efficiency $\eta = g/\Delta B \approx 8 \times 10^{-15}$.

\subsubsection{GitHub Copilot}
\textbf{B-IQ Estimate}: $\Delta B \approx 5 \times 10^2$ nat/s. Miller \& Buschman \cite{miller2015}. establish working memory capacity of 4±1 items with processing limitations due to oscillatory brain rhythms. Individual programming session: developer sensory input (screen) $\sim 10^6$ bit/s, but focused attention is limited by working memory to $\sim$4 simultaneous items. Assuming each programming concept requires $\sim$25 bits of information and updates occur at $\sim$1 Hz: effective bandwidth $\approx 4 \times 25 \times 1 \approx 10^2$ bit/s \cite{miller2015}. Copilot suggestions add $\sim 10^2$ nat/s predictive information about code completion patterns, effectively augmenting human cognitive bandwidth.

\textbf{Efficiency}: Peng et al. \cite{peng2023} report 55.8\% faster task completion in controlled trials. Converting to instantaneous growth rate: if development time decreases by factor $f = 0.558$, this implies a time advantage ratio of $1/(1-f) = 1/0.442 \approx 2.26$. For tasks completed in characteristic time $\tau \sim 1$ hour, the growth rate is $g = \ln(2.26)/\tau \approx 0.82/3600 \approx 2.3 \times 10^{-4}$ s$^{-1}$. Additional studies \cite{github2023, ziegler2024, pandey2024} report 30-50\% time savings across various coding tasks. Using the conservative 55.8\% figure: $g \approx 2.3 \times 10^{-4}$ s$^{-1}$. Efficiency $\eta = g/\Delta B \approx 4.6 \times 10^{-7}$.

\subsubsection{Kiva Warehouse}
\textbf{B-IQ Estimate}: $\Delta B \approx 10^9$ nat/s. Fleet of $\sim 10^3$ robots, each with sensor bandwidth $\sim 10^6$ bit/s (LiDAR, cameras), actuator capacity $\sim 10^4$ bit/s (motion control) \cite{wulfraat2012}. Central coordination system processes $\sim 10^9$ nat/s for optimization and routing. Robots achieve 2-3x productivity gains over manual picking through goods-to-person paradigm.

\textbf{Efficiency}: Operational efficiency gain $g \approx 2.2 \times 10^{-8}$ s$^{-1}$ from throughput improvements vs traditional warehouses \cite{wulfraat2012}. Pick rates of 600+ lines/hour vs 100-180 lines/hour manual operations. Efficiency $\eta = g/\Delta B \approx 2 \times 10^{-17}$.

\subsubsection{Large US Firms}
\textbf{B-IQ Estimate}: $\Delta B \approx 2 \times 10^{12}$ nat/s. Enterprise with $\sim 10^5$ employees, each contributing $\sim 10^7$ nat/s effective information processing (accounting for coordination overhead). Information systems, databases, and communication infrastructure scale total capacity to $\sim 10^{12}$ nat/s. Large firms show systematic productivity advantages over smaller firms across industries \cite{leung2008, haltiwanger1999}.

\textbf{Efficiency}: Canadian data \cite{leung2008, haltiwanger1999} show large firms (>500 employees) achieve 2-3x higher labor productivity than small firms (<100 employees). Assuming similar US patterns and annual productivity advantage of $\sim$3\%, we estimate growth differential $g \approx \ln(1.03)/(365 \times 24 \times 3600) \approx 9.5 \times 10^{-10}$ s$^{-1}$ between large and small firms. This reflects economies of scale in human capital, technology adoption, and market access. Efficiency $\eta = g/\Delta B \approx 5 \times 10^{-22}$.

\subsubsection{Microservice Guilds}
\textbf{B-IQ Estimate}: $\Delta B \approx 10^3$ nat/s. Small cluster ($N = 10$--$10^2$) of stateless services. Each service processes $\sim 10^2$ requests/s with $\sim 10$ bit/request metadata. Limited coordination overhead due to loose coupling: total B-IQ $\approx N \times 10^1 \approx 10^3$ nat/s.

\textbf{Efficiency}: Development velocity metric $\eta \approx 10^{-9}$ from deployment frequency improvements.

\subsubsection{Kubernetes Clusters}
\textbf{B-IQ Estimate}: $\Delta B \approx 10^6$ nat/s. Medium-scale orchestration ($N = 10^3$--$10^4$ pods). Container scheduling, service discovery, and resource management require significant coordination overhead. Orchestration plane processes $\sim 10^6$ events/s, each carrying $\sim 1$ nat of scheduling information.

\textbf{Efficiency}: Operational efficiency $\eta \approx 10^{-14}$ reflecting coordination overhead costs.

\subsubsection{Multi-Tenant Platforms}
\textbf{B-IQ Estimate}: $\Delta B \approx 10^9$ nat/s. Large-scale cloud platform ($N = 10^5$--$10^6$ instances). Service mesh and load balancing enable higher coordination efficiency. Per-instance B-IQ $\sim 10^3$ nat/s with coordination factor $\rho \approx 0.1$.

\textbf{Efficiency}: Platform economies of scale yield $\eta \approx 10^{-16}$.

\subsubsection{Global AI Society}
\textbf{B-IQ Estimate}: $\Delta B \approx 10^{12}$ nat/s. Internet-scale AI coordination ($N = 10^7$--$10^8$ devices). IoT sensors, robotic actuators, and distributed AI services. Theoretical maximum limited by physical communication latencies and coordination protocols.

\textbf{Efficiency}: Projected efficiency $\eta \approx 10^{-15}$ assuming optimal coordination infrastructure.

\subsubsection{Human Individual}
\textbf{B-IQ Estimate}: $\Delta B \approx 3 \times 10^5$ nat/s. Miller \& Buschman \cite{miller2015} establish working memory capacity of 4±1 items constrained by oscillatory brain rhythms. Human cortex has $\sim 10^{11}$ neurons firing at $\sim 10$ Hz, yielding theoretical capacity $\sim 10^{12}$ bit/s. However, conscious processing is severely limited: only $\sim$4 items can be held simultaneously, updated at $\sim$10 Hz with $\sim$16 bits/item (2$^{16} \approx$ 65,000 concepts), giving effective bandwidth $\approx 4 \times 10 \times 16 \approx 6 \times 10^2$ bit/s for conscious thought. For goal-directed behavior involving subconscious processing, we estimate $\sim 10^5$ nat/s \cite{miller2015}. Miller (1956) originally suggested 7±2 items, but modern estimates favor $\sim$4.

\textbf{Efficiency}: Individual baseline—no growth differential computed.

\subsubsection{Bands/Tribes}
\textbf{B-IQ Estimate}: $\Delta B \approx 10^6$ nat/s. Small social group ($N = 10$--$10^2$) with high trust and direct communication. Minimal coordination overhead due to kin networks. Total B-IQ $\approx N \times 3 \times 10^5 / 10^2 \approx 10^6$ nat/s accounting for information sharing benefits.

\textbf{Efficiency}: Social cooperation benefits yield $\eta \approx 10^{-11}$.

\subsubsection{Chiefdoms}
\textbf{B-IQ Estimate}: $\Delta B \approx 10^8$ nat/s. Mid-scale society ($N = 10^3$--$10^4$) with hierarchical organization. Coordination overhead increases due to status competition and reduced trust. Effective per-capita B-IQ drops to $\sim 10^4$ nat/s.

\textbf{Efficiency}: Organizational friction yields minimum efficiency $\eta \approx 10^{-17}$.

\subsubsection{Nation-States}
\textbf{B-IQ Estimate}: $\Delta B \approx 10^{11}$ nat/s. Large-scale society ($N = 10^5$--$10^8$) with institutional infrastructure. Legal systems, markets, and communication networks enable coordination recovery. Per-capita effective B-IQ $\sim 10^3$ nat/s due to institutional overhead.

\textbf{Efficiency}: Institutional coordination yields $\eta \approx 10^{-14}$.

\subsubsection{GPT-4 Inference}
\textbf{Current B-IQ}: $\sim 10^8$ nat/s. Transformer model processing $\sim 10^3$ tokens/s during inference \cite{goel2024}, each token carrying $\sim 10^5$ bits of semantic information. Attention mechanisms and feed-forward layers compress and transform this to $\sim 10^8$ nat/s effective information processing. Performance varies significantly with inference framework, with TensorRT-LLM achieving highest throughput \cite{erdil2024}.

\textbf{Theoretical Max}: $\sim 10^{13}$ nat/s. Hardware substrate (GPUs) capable of $\sim 10^{15}$ operations/s, but model architecture and memory bandwidth limit effective B-IQ to $\sim 10^{13}$ nat/s. Network latency becomes critical bottleneck for distributed inference.

\textbf{Utilization}: Current systems achieve $\sim 10^{-5}$ of theoretical maximum, limited by model inefficiencies and coordination overhead \cite{erdil2024}.

\subsubsection{Human Expert}
\textbf{Current B-IQ}: $\sim 10^6$ nat/s. Expert-level performance in focused domains (medicine, chess, programming) extends beyond the basic 4-item working memory limit \cite{miller2015} through chunking and domain-specific pattern recognition. Experts process complex patterns as single chunks, effectively increasing functional capacity from $\sim 10^2$ bit/s (novice) to $\sim 10^6$ nat/s through practiced access to long-term memory structures and automated pattern recognition.

\textbf{Theoretical Max}: $\sim 10^{11}$ nat/s. Full cortical capacity with optimal information integration across all brain regions.

\textbf{Utilization}: Experts achieve $\sim 10^{-5}$ utilization through focused attention and domain knowledge.

\subsubsection{Warehouse Robotics}
\textbf{Current B-IQ}: $\sim 10^9$ nat/s. Modern automated warehouse with $\sim 10^3$ robots, each processing $\sim 10^6$ bit/s sensory input and $\sim 10^4$ bit/s motor control. Central coordination system adds $\sim 10^9$ nat/s for optimization and routing.

\textbf{Theoretical Max}: $\sim 10^{13}$ nat/s. Physical limits from sensor bandwidth ($\sim 10^{12}$ bit/s aggregate) and actuator capacity ($\sim 10^{11}$ bit/s) across the facility.

\textbf{Utilization}: Current systems achieve $\sim 10^{-4}$ efficiency due to conservative control algorithms and safety margins.

\subsubsection{National Economy}
\textbf{Current B-IQ}: $\sim 10^{11}$ nat/s. GDP-scale information processing across $\sim 10^8$ economic agents, financial networks, supply chains, and government institutions. Each agent contributes $\sim 10^3$ nat/s effective economic decision-making capacity.

\textbf{Theoretical Max}: $\sim 10^{15}$ nat/s. Theoretical limit if all communication channels, computing infrastructure, and human cognitive capacity were optimally coordinated for economic information processing.

\textbf{Utilization}: National economies achieve $\sim 10^{-4}$ efficiency due to market inefficiencies, institutional friction, and coordination problems.

%------------------------------------------------------------------
\section{Hierarchical Aggregation}
A multi-layer agent stack \(\{\mathcal B^{(k)}\}\) with actuator weight
\(\omega_k=C_{\mathrm{act}}^{(k)}/\sum_j C_{\mathrm{act}}^{(j)}\) obeys
\begin{equation}
\mathrm{B\!-\!IQ}_{\text{hier}}
=\sum_k\omega_k\Bigl[I_{\mathrm{pred}}^{(k)}
+\alpha I_{\mathrm{ctrl}}^{(k)}
-\beta H(I^{(k)})
-\gamma S^{(k)}\Bigr].
\end{equation}
When coordination lifts \(\rho\) and amortises cost, the top-level B-IQ asymptotically approaches the sensor/actuator ceiling (Sec.\,\ref{sec:bounds}).

%------------------------------------------------------------------
\section{Scaling Pattern}
\label{sec:scaling}

\subsection{Scaling Ansatz}
\label{sec:ansatz}

We model efficiency scaling using power laws for system parameters. For an $N$-unit system, assume:

\begin{align}
b(N) &\propto N^a, & p(N) &\propto N^{-b}, \nonumber\\
\rho(N) &\approx \frac{1}{1+e^{-\kappa(N-N_0)}}, & c(N) &\propto N^c, \nonumber\\
\Delta B(N) &\propto N^{B_1} \text{ (per-unit } B_1\text{).}
\end{align}

Then efficiency scales as:
\begin{equation}
\eta(N) = \frac{g}{\Delta B(N)} \propto \frac{b(N)p(N)\rho(N)}{c(N)N^{B_1}} \propto N^{a-b-c-1}\rho(N).
\label{eq:scaling}
\end{equation}

\subsection{AI Hierarchy Scaling}
\label{sec:ai_scaling}

Table \ref{tab:ai_scaling} details scaling regimes across AI system layers:

\begin{table}[!ht]
\centering\footnotesize
\caption{AI hierarchy scaling parameters}
\label{tab:ai_scaling}
\begin{tabular}{lcc}
\hline
Layer & Scale ($N$) & Exponent regime \\
\hline
Microservice guild & $10$--$10^2$ & $a=0, b=1, c=0.5$ \\
Kubernetes cluster & $10^3$--$10^4$ & $\rho$ rising \\
Multi-tenant platform & $10^5$--$10^6$ & $\rho \to 1$ \\
Global AI society & $10^7$--$10^8$ & $c$ amortized \\
\hline
\end{tabular}
\end{table}

\textbf{Initial drop} ($N < N_0$): $\eta \sim N^{-2.5}$ plummets as microservices fragment.

\textbf{Rebound} ($N \gtrsim N_0 \approx 10^4$): $\rho \to 1$ and if $a-c-1 > 0$ (e.g., $c \approx 0.5$), then $\eta \propto N^{-1.5}$ or flattens, recovering efficiency.

\textbf{Ceiling}: Sensor/actuator channels saturate at $C_{\text{sens}}, C_{\text{act}}$, capping $b$ and $c$.

\subsection{Human Hierarchy Scaling}
\label{sec:human_scaling}

Table \ref{tab:human_scaling} presents scaling across human organizational layers:

\begin{table}[!ht]
\centering\footnotesize
\caption{Human hierarchy scaling parameters}
\label{tab:human_scaling}
\begin{tabular}{lcc}
\hline
Layer & Scale ($N$) & Exponent regime \\
\hline
Individual & $1$ & -- \\
Bands/tribes & $10^1$--$10^2$ & $a \approx 0.5, b \approx 1, c \approx 1$ \\
Chiefdoms & $10^3$--$10^4$ & $\rho$ dips \\
Nation-states & $10^5$--$10^8$ & $\rho \to 1$ \\
\hline
\end{tabular}
\end{table}

\textbf{Small scale}: $\eta \sim N^{-2.5}$ but high $\rho$ softens drop ($N < 10^2$).

\textbf{Mid scale}: Chiefdoms see minimal $bp\rho/c$, yielding lowest $\eta$.

\textbf{Macro scale} ($N > 10^5$): Legal/institutional scaffolding drives $\rho \to 1$ and amortizes $c$, so $\eta \propto N^{-1.5}$ or better.

\subsection{Universal Scaling Principles}
\label{sec:universal}

Both AI and human hierarchies exhibit:

1. \textbf{Fragmentation penalty}: Small-scale systems achieve high per-bit efficiency only up to threshold $N_0$.

2. \textbf{Mid-range collapse}: Without coordination technology, overheads $c$ and low $p$, $\rho$ cause $\eta$ to drop by orders of magnitude.

3. \textbf{Institutional rebound}: Once rules/protocols bind many units ($N > N_0$), $p$ and $\rho$ rise while $c$ is shared, partially restoring $\eta$.

4. \textbf{Convergent ceilings}: Both hierarchies approach the same physical IO limits ($C_{\text{sens}}, C_{\text{act}}$), giving comparable ultimate competence bounds ($\sim 10^{13}$--$10^{14}$ nat s$^{-1}$).

\subsection{Sensitivity to \texorpdfstring{$\alpha,\beta,\gamma$}{(alpha,beta,gamma)}}
\label{sec:sens}
Using the anchors of Table~\ref{tab:anchors} we test how varying the coefficients shifts
\(\mathrm{B\!-\!IQ}\) and the derived efficiency
\(\eta=g/\!\Delta B\).

\begin{itemize}
\item \textbf{Control scaling $(\alpha)$.}  Across anchors
\(I_{\mathrm{ctrl}}\!/\!I_{\mathrm{pred}}\in[0.02,0.3]\).
Replacing $\alpha=1$ by $\alpha\in[0.5,2]$ alters
\(\mathrm{B\!-\!IQ}\) by
$<20\%$ (humans) and $<5\%$ (LLMs, firms); efficiency brackets in
Sec.~\ref{sec:anchors}
shift by $\le 0.2$ decades.

\item \textbf{Surprise scaling $(\gamma)$.}  
Observed $S/I_{\mathrm{pred}}\le0.2$.  
Choosing $\gamma\in[0.5,2]$ therefore perturbs total B-IQ by at most
$\pm20\%$, leaving the rank order of agents unchanged.

\item \textbf{Memory penalty $(\beta)$.}  
If $\beta$ were raised from $10^{-3}$ to $1$, the B-IQ of silicon
megaclusters (dominated by $H(I)$) would fall two orders, possibly
dropping below human–society levels.  
All qualitative statements in \S\ref{sec:bounds}–\ref{sec:scaling}
remain valid provided $\beta\!<\!10^{-2}$; above that threshold
memory cost becomes decisive and large-memory agents lose
their advantage.
\end{itemize}

\paragraph*{Summary}
For physically grounded settings
\(\alpha=\gamma=1,\;\beta\!\ll\!1\),
uncertainty in the weights modifies numerical B-IQ values by
$\lesssim 0.3$ orders—small relative to the 10–15-order span seen in
the empirical $\eta$ distribution.
The tentative U-shape in efficiency is therefore
unlikely to be an artefact of coefficient choice, though it may still
reflect sampling bias or hidden covariates.

%------------------------------------------------------------------
\section{Absolute Limits and Current Position}
\label{sec:limits}

\subsection{Theoretical Ceilings}
\label{sec:ceilings}

Physical constraints impose hard bounds on achievable B-IQ:

\begin{align}
\text{B-IQ}_{\max} &= C_{\text{sens}} + \alpha C_{\text{act}} - \beta C_{\text{mem}}^{\text{rt}} - \gamma S_{\min} \nonumber\\
&\leq C_{\text{sens}} + C_{\text{act}} \quad \text{(for } \beta, \gamma S_{\min} \ll C_{\text{sens}}\text{)}
\end{align}

For concrete systems:
\begin{itemize}
\item \textbf{10 MW datacenter}: Optical fiber ingress $\sim 10^{12}$ bit/s, robotic actuators $\sim 10^{11}$ bit/s, yielding $\text{B-IQ}_{\max} \sim 10^{14}$ nat/s.
\item \textbf{Human cortex}: Neural bandwidth $\sim 10^{12}$ spike/s $\times$ 1 bit/spike $\approx 10^{11}$ nat/s.
\item \textbf{Global internet}: Aggregate bandwidth $\sim 10^{15}$ bit/s, but coordination overhead limits effective B-IQ to $\sim 10^{12}$ nat/s.
\end{itemize}

\subsection{Distance from Limits}
\label{sec:distance}

Current systems operate far below theoretical maxima:

\begin{table}[!ht]
\centering\footnotesize
\caption{Distance from theoretical limits}
\label{tab:limits}
\begin{tabular}{lccc}
\hline
System & Current B-IQ & Theoretical Max & Utilization \\
\hline
GPT-4 inference & $\sim 10^8$ & $\sim 10^{13}$ & $\sim 10^{-5}$ \\
Human expert & $\sim 10^6$ & $\sim 10^{11}$ & $\sim 10^{-5}$ \\
Warehouse robotics & $\sim 10^9$ & $\sim 10^{13}$ & $\sim 10^{-4}$ \\
National economy & $\sim 10^{11}$ & $\sim 10^{15}$ & $\sim 10^{-4}$ \\
\hline
\end{tabular}
\end{table}

\textbf{Key findings:}
1. \textbf{Massive headroom}: Even advanced systems utilize $<10^{-3}$ of their theoretical capacity.
2. \textbf{Coordination bottleneck}: Large systems are limited by $\rho(N)$ rather than raw processing power.
3. \textbf{Efficiency opportunity}: Orders of magnitude improvement possible through better coordination protocols.

\subsection{Scaling Trajectories}
\label{sec:trajectories}

Extrapolating current trends:
\begin{itemize}
\item \textbf{AI systems}: Following $N^{-1.5}$ scaling, global AI society ($N \sim 10^8$) could achieve $\eta \sim 10^{-15}$ with $\rho \to 1$.
\item \textbf{Human institutions}: Nation-states approach $\sim 10^{-14}$ efficiency ceiling, limited by communication latency.
\item \textbf{Hybrid systems}: AI-human collaboration may overcome individual limitations, potentially reaching $\eta \sim 10^{-12}$.
\end{itemize}

%------------------------------------------------------------------
\section{Discussion}

\textbf{Hierarchical universality.} The U-shaped efficiency scaling observed across both AI and human hierarchies suggests fundamental constraints on information processing at different organizational scales. 
The mid-range efficiency minimum around $N \sim 10^{3}$--$10^{4}$ appears to be a universal coordination bottleneck where systems are too large for direct communication but too small for effective institutional structures. 
Notably, this pattern may help explain why empires and large organizations often experience internal collapse: 
As they grow beyond optimal coordination scales, efficiency drops precipitously, making them vulnerable to smaller, more agile competitors or internal fragmentation.

\textbf{Physical bounds and headroom.} Current systems operate 3--5 orders of magnitude below their theoretical limits. Even a 10 MW datacenter, saturating both optical ingress and robotic egress, tops out near $10^{14}$ nat s$^{-1}$—roughly 500× a whole human cortex. This massive headroom suggests that competence growth is limited by coordination protocols rather than raw computational capacity.

\textbf{Efficiency patterns.} Biological micro-agents convert competence into fitness at $\eta \sim 10^{-12}$--$10^{-14}$; current industrial systems sit many orders lower, suggesting large slack. The efficiency rebound at macro scales ($N > 10^5$) in both AI and human systems indicates that institutional scaffolding can partially overcome coordination overhead, though never fully recovering small-scale efficiency.

\textbf{Convergent ceilings.} Both AI and human hierarchies approach similar ultimate competence bounds ($\sim 10^{13}$--$10^{14}$ nat s$^{-1}$), suggesting that environment complexity and coordination infrastructure, rather than substrate differences, determine system-level intelligence limits.

\textbf{Implications for AI development.} The analysis suggests that building larger AI systems without addressing coordination efficiency ($\rho(N)$) may yield diminishing returns. Focus should shift from raw model scaling to developing better protocols for multi-agent coordination and information sharing. The cooperativity index $\kappa_{ij}$ (Sec.~\ref{sec:cooperativity}) provides a quantitative framework for optimizing transparency versus privacy in multi-agent AI systems, with percolation thresholds determining when cooperative behaviors spread globally versus remaining localized.

\textbf{Caveats.} Anchors use heterogeneous proxies; memory costs $\beta H(I)$ are approximated; sensor entropy estimates are coarse. The observed U-shape may reflect sampling bias across different temporal and spatial scales. Larger datasets with more uniform measurement protocols are needed to confirm these scaling patterns.

\textbf{Future work.} We intend to develop a common measurement protocol for B-IQ across diverse systems. Current estimates rely on heterogeneous methodologies, making direct comparisons difficult. Standardized approaches for measuring information processing capacity, growth rates, and efficiency would enable more robust cross-system analysis and better validation of the proposed scaling laws.

%------------------------------------------------------------------
\section{Conclusion}
B-IQ provides a principled, language-free gauge of agent competence derivable from blanket statistics.  Preliminary anchors span 15 orders of efficiency, hinting that coordination overhead—not raw information processing—dominates large-scale performance.  Further empirical work is required to confirm or refute the tentative U-shape in \(\eta(N)\).

%------------------------------------------------------------------
\bibliographystyle{IEEEtran}
\begin{thebibliography}{25}
\bibitem{Tishby2000IB} N.~Tishby, F.~Pereira, and W.~Bialek, ``The information bottleneck method,'' \emph{Proc.\ Allerton}, 2000.

% UAD and cooperation papers
\bibitem{uad2025} L.~et al., ``Foundations of unsupervised agent discovery in raw dynamical systems,'' 2025.
\bibitem{abc2025} G.~Zarncke, ``Attractor basins of cooperation, privacy, and parasite persistence: Applications of unsupervised agent discovery,'' 2025.

% Ant sensory and motor capabilities
\bibitem{seidl2008} T.~Seidl and R.~Wehner, ``Walking on inclines: how do desert ants monitor slope and step length,'' \emph{Front.\ Zool.}, vol.~5, no.~8, 2008.
\bibitem{dauzere2024} O.~Dauzere-Peres and A.~Wystrach, ``Ants integrate proprioception as well as visual context and efference copies to make robust predictions,'' \emph{Nat.\ Commun.}, vol.~15, no.~10205, 2024.
\bibitem{gordon2020} D.~M.~Gordon, ``Movement, encounter rate, and collective behavior in ant colonies,'' \emph{Ann.\ Entomol.\ Soc.\ Am.}, vol.~114, no.~5, pp.~541--546, 2020.

% Fire ant colony dynamics
\bibitem{tschinkel2011} W.~R.~Tschinkel, ``The organization of foraging in the fire ant, Solenopsis invicta,'' \emph{J.\ Insect Sci.}, vol.~11, no.~26, 2011.
\bibitem{ant_nest_rate} J.~Drescher \emph{et al.}, ``Nest productivity of Formica spp.,'' \emph{Ecol.\ Entomol.}, 2020.
\bibitem{pharaoh_growth} S.~Heinze, ``Colony growth dynamics in Pharaoh ants,'' \emph{Insectes Soc.}, 2019.

% GitHub Copilot productivity studies
\bibitem{github2023} GitHub, ``Research: Quantifying GitHub Copilot's impact on code quality,'' GitHub Blog, Oct.~2023.
\bibitem{ziegler2024} A.~Ziegler \emph{et al.}, ``Measuring GitHub Copilot's impact on productivity,'' \emph{Commun.\ ACM}, vol.~67, no.~3, pp.~54--63, 2024.
\bibitem{peng2023} S.~Peng, E.~Kalliamvakou, P.~Cihon, and M.~Demirer, ``The impact of AI on developer productivity: Evidence from GitHub Copilot,'' \emph{arXiv:2302.06590}, 2023.
\bibitem{pandey2024} R.~Pandey, P.~Singh, R.~Wei, and S.~Shankar, ``Transforming software development: Evaluating the efficiency and challenges of GitHub Copilot in real-world projects,'' \emph{arXiv:2406.17910}, 2024.
\bibitem{goel2024} A.~Goel and R.~Borgohain, ``Exploring LLMs speed benchmarks: Independent analysis,'' Inferless Blog, 2024.

% Warehouse robotics
\bibitem{kiva2014} P.~Wurman \emph{et al.}, ``Optimizing warehouse automation with Kiva robots,'' \emph{AI Mag.}, 2014.
\bibitem{wulfraat2012} M.~Wulfraat, ``Kiva Systems: A supply chain consultant evaluation,'' MWPVL International White Paper, 2012.

% Economic data
\bibitem{leung2008} D.~Leung, C.~Meh, and Y.~Terajima, ``Firm size and productivity,'' Staff Working Papers 08-45, Bank of Canada, 2008.
\bibitem{haltiwanger1999} J.~C.~Haltiwanger, J.~I.~Lane, and J.~Spletzer, ``Productivity differences across employers: The roles of employer size, age, and human capital,'' \emph{Am.\ Econ.\ Rev.}, vol.~89, no.~2, pp.~94--98, 1999.

% Human cognitive limits
\bibitem{miller2015} E.~K.~Miller and T.~J.~Buschman, ``Working memory capacity: Limits on the bandwidth of cognition,'' \emph{Daedalus}, vol.~144, no.~1, pp.~112--122, 2015.
\bibitem{miller1956} G.~A.~Miller, ``The magical number seven, plus or minus two: Some limits on our capacity for processing information,'' \emph{Psych.\ Rev.}, vol.~63, no.~2, pp.~81--97, 1956.

% LLM inference performance
\bibitem{erdil2024} E.~Erdil, ``Inference economics of language models,'' arXiv preprint arXiv:2506.04645, 2024.

\end{thebibliography}

\end{document}
