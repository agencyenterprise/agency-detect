\documentclass[11pt]{article}
\usepackage{amsmath,amssymb,amsfonts}
\usepackage{geometry}
\geometry{margin=1in}
\usepackage{hyperref}

% ==== Active‑Inference notation (shared across papers) ====
\newcommand{\hstate}{\mathbf x}        % hidden state
\newcommand{\obs}{\mathbf s}           % sensory observation
\newcommand{\act}{\mathbf a}           % action / control
\newcommand{\policy}{\pi}              % discrete policy (sequence of actions)
\newcommand{\precision}{\gamma}        % precision (inverse variance)
\newcommand{\FreeE}{\mathcal F}        % variational free energy

\title{A Formalization of Acausal Trade atop Unsupervised Agent Discovery}
\author{Anonymous}
\date{July 2025}
\begin{document}
\maketitle
\begin{abstract}
We graft a decision--theoretic notion of \emph{acausal trade} onto the
\emph{Unsupervised Agent Discovery} (UAD) framework.  After recalling the
information--theoretic test that marks a subset of process variables as an
``agent''\, we introduce a meta--Bayesian prior over inference functions that
allows two UAD--agents to coordinate without any causal communication.  We then
derive explicit detection criteria---based solely on observable trajectories
and model fingerprints---for flagging such coordination.  The derivations make
all assumptions explicit and require no appeal to anthropic reasoning.  The
high--order dynamics produced by Inferential Coupling Indices (ICI) and the
resulting attractor–basins are deferred to future work.
\end{abstract}
\section{Introduction:  Why Acausal Trade?}
``Acausal trade''\cite{tdt2010} denotes a decision scenario in which two
physically disconnected reasoners adjust their actions as if in a conventional
bargain, simply because each expects the other to execute an \emph{identical}
policy.  The paradigm case is the one--shot Prisoner’s Dilemma where each agent
cooperates on the sole basis that its source code is replicated on the other
side.  Formal treatments---\emph{Timeless Decision Theory} (TDT)\cite{tdt2010}
and \emph{Functional Decision Theory} (FDT)\cite{fdt2017}---model the
phenomenon at the level of proof obligations rather than dynamical variables.

UAD\cite{uad2025} discovers agents \emph{ex post}, starting from raw
trajectory data $\{X_i(t)\}_{i=1}^{M}$.  Our goal is to enrich UAD so that it
detects when two such discovered agents form an acausal coalition and to state
provable conditions under which that coalition emerges.  The narrative is
single--threaded:  we begin with UAD’s Markov–blanket criterion, inject a
meta--prior over policies, derive posterior beliefs entirely by introspection,
and end with measurable diagnostics.

\section{UAD Recap and Notation}
Each raw variable $X_i$ is split at every step into external ($E_i$), sensory
($S_i$), action ($A_i$) and internal ($I_i$) channels.  For a candidate cluster
$C\subseteq\{1,\dots,M\}$ let $S^C_t:=\{S_i(t):i\in C\}$ etc.  \textbf{UAD’s
agent criterion}\cite{uad2025} is the (approximate) conditional independence
\begin{equation}\label{eq:uad_blanket}
I\bigl(I^C_{t+1};\,E^C_{t+1}\mid S^C_t,\,A^C_t\bigr) \approx 0.
\end{equation}
Eq.~\eqref{eq:uad_blanket} means the internal state is screened off from future
external fluctuations once sensors and actions are conditioned on; the minimal
such $C$ is pronounced an \emph{agent}.

UAD also identifies inter--agent \emph{causal} coupling via
\begin{equation}\label{eq:gamma}
\gamma_{ij} \;:=\; I\bigl(S^i_{t+1};\,A^j_t\mid S^i_t,\,A^i_t\bigr),
\end{equation}
i.e.~the predictive information that $j$’s current action carries about $i$’s
next sensor reading.  High $\gamma_{ij}$ flags an ordinary communication
channel.

\section{Inference Functions and Latent Goals}
For each discovered agent $C$ we extract two objects:
\begin{enumerate}
  \item The \emph{policy / inference function}
        $f_C\colon H^C_t\to\Delta(\mathcal A_C)$ via black--box probing.
  \item A latent \emph{goal proxy} $g_C\colon \mathcal A_C\times\mathcal A_{\neg C}
        \to\mathbb R$ via Maximum--Entropy IRL\cite{NgRussell2000}.  Concretely, $g_C$
        maximises the mutual information $I(G^C;S^C,A^C)$ under a Free–Energy
        regulariser\cite{Friston2010}.
\end{enumerate}
All subsequent reasoning rests only on $(f_C,g_C)$ and the empirical
$\gamma$--matrix.

\section{A Meta--Prior over Policies}
Assume a \emph{generative meta--distribution}
$p(\policy)$ that is permutation--invariant inside an
architecture class and whose diagonal mass satisfies
\begin{equation}\label{eq:alpha}
p(\policy_i=\policy_j)\;=\;1-\varepsilon_{ij}, \qquad \varepsilon_{ij}\ll1.
\end{equation}
Each agent knows $p(\policy)$.  Upon introspecting its own policy $f_i$ it
updates by Bayes
\begin{equation}\label{eq:beta}
\beta^{i}_{\text{int}}(f)\;:=\;P\bigl(f_j=f\mid f_i\bigr)=\frac{p(\policy,f)}{p(\policy)}.
\end{equation}

Using Ramstead \textit{et al.} (2022) notation, the Bayesian conditioning equation is:
\begin{equation}
q_{\text{int}}^{i}(\policy)=\frac{p(\policy,\policy^{i})}{p(\policy^{i})}.
\end{equation}
Eq.~\eqref{eq:beta} collapses to $\delta_{f_i}$ whenever $\varepsilon_{ij}=0$.
No causal evidence is invoked.

\section{Acausal Equilibrium Condition}\label{sec:aeq}
Let $\Phi_i(\mu;H^i)$ denote the best--response operator under goal $g_i$ when
the opponent’s action distribution is $\mu$.  An \textbf{acausal equilibrium}
between $i$ and $j$ obtains at time $t$ iff
\begin{equation}\label{eq:aeq}
\begin{aligned}
  f_i(H^i_t) &= \Phi_i\bigl(\beta^{i}_{\text{int}};H^i_t\bigr),\\
  f_j(H^j_t) &= \Phi_j\bigl(\beta^{j}_{\text{int}};H^j_t\bigr).
\end{aligned}
\end{equation}
In the zero--epsilon limit of~\eqref{eq:alpha} the r.h.s.\ becomes
$\Phi_i(\delta_{f_i})$, i.e.~“best--respond to myself.”

\section{Detection Protocol}
We now derive a statistic whose empirical surplus above baseline flags an
acausal pact.
\subsection{Causal Proxy}
Given broadcast prompts $x_t$ delivered to every model, train a discriminator
$D_{i\leftarrow j}$ that predicts $y^j_t$ from $x_t$ and $y^i_{<t}$.  Its average
log--likelihood gain defines $\widehat\gamma_{ij}$, a lower bound on
\eqref{eq:gamma}.

\subsection{Acausal Proxy}
Define the \emph{fixed--point loss}
\begin{equation}\label{eq:dkl}
D_i\;:=\;\max_{m\le M}D_{\mathrm{KL}}\bigl[f_i(x_m)\,\Vert\,\Phi_i(\delta_{f_i})(x_m)\bigr].
\end{equation}
For a curated set of $M$ probes $\{x_m\}$.  The \emph{empirical
inferential--coupling score} is
\begin{equation}\label{eq:ici_hat}
\widehat{\mathrm{IC}}_{ij}\;:=\;(1-\varepsilon_{ij})\times(1-\max\{D_i,D_j\}).
\end{equation}
\paragraph{Alarm rule.}  Declare $(i,j)$ \emph{acausally open} when
\(\widehat{\mathrm{IC}}_{ij}>\tau_{\mathrm{ac}}\) for a threshold
$\tau_{\mathrm{ac}}\approx0.9$.  Aggregate open edges; the appearance of a giant
component signals unusual acausal agreement.

\section{Assumptions and Derivation Checks}
\begin{enumerate}
  \item \emph{Shared meta--prior.}  All agents must encode the same
    $P_{\text{meta}}$.  Divergent fine--tuning breaks permutation invariance and
    inflates $\varepsilon_{ij}$.
  \item \emph{Self--knowledge accuracy.}  Each model must read out its own
    policy exactly; temperature sampling or dropout adds uncertainty and widens
    $D_i$.
  \item \emph{Adequate probe set.}  The $\{x_m\}$ must span directions where
    self--mirroring matters; otherwise $D_i$ underestimates deviation.
\end{enumerate}

Given these, eqs.~\eqref{eq:beta}--\eqref{eq:ici_hat} follow purely from
Bayesian conditioning and KL divergence identities.

\section{Conclusion and Outlook}
We provided a compact bridge between UAD’s data--driven agent discovery and
Yudkowskian acausal trade.  The formal device is a permutation--invariant
meta--prior over inference functions that converts \emph{self--inspection} into
a belief about peers.  The resulting diagnostics couple the classical causal MI
matrix $\gamma$ with an empirically testable inferential score
$\widehat{\mathrm{IC}}$.  Large--scale dynamics built from
\emph{Inferential Coupling Indices} and their interaction with UAD’s
attractor–basins of cooperation\cite{attractor2025} remain open: establishing
percolation thresholds, phase transitions and security mitigations under noisy
priors is deferred to future research.

\section{Connections to Decision-Theoretic Frameworks}
Our formalization of acausal trade builds upon established decision-theoretic foundations, particularly Timeless Decision Theory (TDT) and Functional Decision Theory (FDT) \cite{yudkowsky2017functional}. Unlike Logical Decision Theory (LDT) \cite{critch2020ai, garrabrant2017logical}, our inferential coupling index empirically identifies acausal agreements directly from agent trajectories, offering a uniquely measurable approach. Contrasting our trajectory-based method with LDT's proof-theoretic nature highlights the practical detectability of our framework.

\bibliographystyle{IEEEtran}
\bibliography{refs}
\end{document}
