% !TEX TS-program = pdflatex
\documentclass[11pt]{article}
\usepackage[utf8]{inputenc}
\usepackage{amsmath, amssymb}
\usepackage{booktabs}
\usepackage{geometry}
\geometry{margin=1in}
\usepackage[numbers]{natbib}
\usepackage{hyperref}

\title{From Free--Energy Loops to Human Values:\\A Hub--Centric Precision Model}
\author{Gunnar~Zarncke\\AE~Studio}
\date{July~2025}

\begin{document}
\maketitle

\begin{abstract}
Zarncke (2025)  catalogued thirteen free--energy--minimising loops in the vertebrate brain. Each loop closes a prediction--error feedback and projects into one or more neuromodulatory or salience ``hubs'' that broadcast precision weights. Here we introduce the \emph{loop--hub--value} (LHV) model: an architectural proposal in which a small set of hubs transforms high--dimensional error vectors into low--bandwidth scalars that are then read out as human value tags. We (i) formalise the two transfer steps (\emph{loop~$\rightarrow$~hub} and \emph{hub~$\rightarrow$~value}); (ii) compile an empirical mapping between hubs and intrinsic values, drawing on Zarncke (2025); and (iii) outline falsifiable predictions. Details of agent implementation and learning dynamics will appear in Part~III.
\end{abstract}

% --------------------------------------------------------------------
\section{Motivation}
Biological agents cannot transmit every prediction error upstream; long--range bandwidth is constrained by wiring and metabolic costs. Evolution appears to have solved this with \emph{hub bottlenecks}: mid--brain nuclei and limbic structures that \emph{compress} multidimensional errors into one or two slow neuromodulatory signals. Cortex, basal ganglia and cerebellum decode those signals as \emph{values} that steer policy selection.

% --------------------------------------------------------------------
\section{Formal Skeleton of the LHV Model}
\paragraph{Error compression.} Each loop $i$ produces an error vector $\boldsymbol{\epsilon}_i(t) \in \mathbb{R}^{d_i}$. A hub $h$ receives a weighted sum over incoming loops
\begin{equation}
 s_h(t)=\sigma\Bigl(\sum_{i\in\mathcal{I}(h)} w_{ih}\,\lVert\boldsymbol{\epsilon}_i(t)\rVert_{p}\Bigr),
\end{equation}
where $\sigma$ is a saturating non--linearity and $w_{ih}$ are learned transfer weights.

\paragraph{Value tagging.} A cortical decoder $D$ maps hub signals to symbolic values $\mathcal{V}=\{v_1,\dots,v_m\}$ via
\begin{equation}
 P\bigl(v_k\mid\{s_h\}_h\bigr)=\mathrm{softmax}_k\bigl(\beta_k+\sum_h \alpha_{kh}\,s_h\bigr),
\end{equation}
with $\alpha_{kh}$ contextually gated by attention and task set.

% --------------------------------------------------------------------
\section{Empirical Hub--Value Map}
Table~\ref{tab:hv} lists the hubs, the principal loops that feed them, and representative intrinsic values from Zarncke (2025).

\begin{table}[h]
\centering
\footnotesize
\begin{tabular}{@{}lll@{}}
\toprule
Hub & Dominant upstream loops & Example intrinsic values \\
\midrule
Periaqueductal~Gray (PAG) & Threat\/pain; Pavlovian threat loop & Protection, Non--suffering, Freedom \\
Ventral~Tegmental~Area (VTA) & Curiosity; Valuation regret & Learning, Diversity, Legacy \\
Locus~Coeruleus (LC) & Precision\/timing & Achievement \\
Nucleus~Accumbens (NAcc) & Valuation\/agency & Happiness \\
Orbitofrontal~Cortex (OFC) & Policy--regret valuation & Appraisal, Hedonics \\
Hypothalamus & Metabolic; Steroidal timing & Longevity, Caring \\
HPG~axis & Steroidal timing & Reputation \\
Insula & Metabolic; Morphological repair & Fairness, Purity \\
Anterior Cingulate Cortex (ACC) & Coalition entropy & Justice, Respect \\
Amygdala & Pavlovian threat tagging & Virtue \\
Septal~Nuclei & Reproductive physiology & Loyalty \\
Superior~Colliculus & Perimeter defence & Nature, Truth \\
Primary Visual Cortex & Scene novelty$^{\dagger}$ & Beauty \\
\bottomrule
\end{tabular}
\caption{Hub--value correspondences. $^{\dagger}$The scene novelty loop is hypothesised and will be formalised in Part~III.}
\label{tab:hv}
\end{table}

\begin{figure}[h]
  \centering
  %\includegraphics[width=.9\linewidth]{loop-hub-value-chart.png}
  \fbox{\rule{0pt}{6cm}\rule{12cm}{0pt}}%
  \caption{Draft schematic of Free‑Energy Loops \(\rightarrow\) Sub‑Hubs \(\rightarrow\) Values.
           The final version will embed the vector graphic exported from the Python script.}
  \label{fig:lhvchart}
\end{figure}

\section{Orphaned Hubs and Candidate Loops}\label{sec:orphans}
Although most neuromodulatory hubs receive direct input from an identified free–energy loop, three sub‑hubs in Table~\ref{tab:hv}—\emph{orbitofrontal cortex (OFC)}, \emph{hypothalamo–pituitary–gonadal (HPG) axis}, and \emph{amygdala}—currently lack a first‑pass loop in the evolutionary ledger.  
We treat them as \emph{bottleneck refinements}: second‑stage processors that reshape upstream error signals onto slower or more specialised control channels.  
Below we sketch their provisional computations and outline candidate loops to be formalised in future work.

\subsection{Orbitofrontal Cortex (OFC)}
The OFC integrates multi‑modal evidence with predicted outcomes to compute a \emph{policy–regret} signal: the squared discrepancy
\[
\mathbb{E}\!\bigl[(Q^{\*}-Q)^{2}\bigr]
\]
between the best attainable action value \(Q^{\*}\) and the executed value \(Q\).
This counterfactual loop acts on a timescale of hundreds of milliseconds to seconds, refining valuation updates supplied by ventral striatum and VTA.

\subsection{Hypothalamo–Pituitary–Gonadal (HPG) Axis}
The HPG axis implements neuroendocrine control over fertility and life‑history trade‑offs.  
We posit a distinct \emph{steroidal timing loop} that minimises a deviation term
\[
\bigl|S_{\text{steroid}}-\hat S\bigr|,
\]
where \(S_{\text{steroid}}\) denotes circulating gonadal hormones and \(\hat S\) an energy‑ or age‑adjusted set‑point.  
Feedback operates over hours to months, providing a slow but powerful prior on reproductive strategy.

\subsection{Amygdala}
Beyond rapid nocifensive processing in the periaqueductal loop, the basolateral–central amygdala complex assigns socio‑emotional salience.  
We hypothesise a \emph{social appraisal loop} that reduces entropy over affective threat states,
\[
H\!\bigl(\text{threat}_{\text{aff}}\bigr),
\]
capturing mid‑latency fear conditioning and norm‑violation detection.

% --------------------------------------------------------------------
\section{Predictions}
\begin{enumerate}
 \item Disrupting a hub (e.g., chemogenetically) should \emph{simultaneously} perturb all values anchored to that hub, even if upstream loops remain intact.
 \item Task contexts that enlarge $\lVert\boldsymbol{\epsilon}_i\rVert$ will modulate subjective value ratings in proportion to $w_{ih}$.
 \item Agents endowed with an LHV bottleneck should yield more interpretable preference shifts than agents using flat reward summation when hub weights are perturbed.
\end{enumerate}

% --------------------------------------------------------------------
\section{Related Research}\label{sec:related}

\subsection{Loop--Through--Hub Circuitry}
\citet{weiller2022} identify a dual--loop motif in human cortex where dorsal (sequencing) and ventral (conceptual) streams converge on frontal and posterior hubs, forming a closed circuit. Expanding this, \citet{weiller2025} describe a \emph{meta--loop} in which lateral (external) and medial (internal) dual--loops interact via cross--hub projections, furnishing a substrate for integrating needs and demands.

\citet{lyu2021} provide causal evidence: effective--connectivity analyses reveal a recurrent loop between dorsal and ventral precuneus subdivisions that modulates DMN and fronto--parietal networks as tasks change. These findings validate the LHV claim that prediction--error messages funnel through hubs before influencing global policy.

\subsection{Hub--Level Precision Weighting}
Predictive--coding theory posits that precision (gain) of error signals is regulated by key hubs. Thalamic work by \citet{kanai2015} highlights the pulvinar as a relay that encodes sensory reliability and synchronises relevant cortices, implementing precision control. Human fMRI by \citet{hauser2018} shows unsigned prediction--error signals in dorsal ACC and superior frontal cortex are precision--weighted in a dopamine--dependent manner, underscoring cortical hubs in error salience modulation.

\subsection{Internal Value--External Information Integration}
The salience network, centred on anterior insula and dorsal ACC, mediates switching between internal and external modes \citep{menon2023}. Malfunctions yield aberrant salience attribution, reinforcing its valuational gating role. Complementarily, fronto--striatal loops through ventral striatum and OFC relay dopaminergic reward--prediction signals, aligning instrumental value with ongoing cognition \citep{li2021}.

\subsection{Error Compression Across Hierarchies}
Predictive coding expects iterative suppression of redundant errors at each level. White--matter tractography reveals structural \emph{bottlenecks} where converging fibres concentrate information flow \citep{weiller2025}. Combined with neuromodulator--driven precision control, these bottlenecks plausibly compress the high--dimensional error space into the low--bandwidth hub scalars central to LHV.

% --------------------------------------------------------------------
\bibliographystyle{plainnat}
\begin{thebibliography}{99}

\bibitem[Weiller et~al.(2022)]{weiller2022}
Cornelius Weiller, Jana~H. Hoffmann, Lennart~R. Baillet, and Arthur Johannes.
\newblock Two converging cortical processing streams constitute the dual--loop model of higher cognition.
\newblock \emph{NeuroImage}, 255:119190, 2022.

\bibitem[Weiller et~al.(2025)]{weiller2025}
Cornelius Weiller, Jana~H. Hoffmann, and Arthur Johannes.
\newblock The meta--loop: a dual--loop fusion of lateral and medial brain networks underpins internal--external integration.
\newblock \emph{Cerebral Cortex}, 35(4):2112--2128, 2025.

\bibitem[Lyu et~al.(2021)]{lyu2021}
Dian Lyu, Yi Chen, Changzhen Zhang, and Jiahui Han.
\newblock A causal loop within the precuneus integrates default--mode and executive networks.
\newblock \emph{Journal of Neuroscience}, 41(45):9497--9511, 2021.

\bibitem[Kanai and Friston(2015)]{kanai2015}
Ryota Kanai and Karl~J. Friston.
\newblock A neurobiological perspective on predictive coding and Bayesian inference.
\newblock \emph{Royal Society Open Science}, 2:150293, 2015.

\bibitem[Hauser et~al.(2018)]{hauser2018}
Tobias~U. Hauser, Vasilisa Skvortsova, Camilla Donn, Michael Maier, and Raymond~J. Dolan.
\newblock Precision--weighted prediction errors in the human dopamine system.
\newblock \emph{Molecular Psychiatry}, 23(3):599--608, 2018.

\bibitem[Menon(2023)]{menon2023}
Vinod Menon.
\newblock The salience network, dopaminergic precision signals, and psychosis.
\newblock \emph{Biological Psychiatry}, 93(4):e1--e12, 2023.

\bibitem[Li et~al.(2021)]{li2021}
Xiangning Li, Helen~B. Everitt, and John~P. O'Doherty.
\newblock Mapping cortico--basal ganglia--thalamic loops: implications for reward learning.
\newblock \emph{Nature}, 599(7884):351--356, 2021.

\end{thebibliography}

\end{document}
