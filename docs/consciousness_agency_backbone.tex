\documentclass[11pt]{article}
\usepackage[T1]{fontenc}
\usepackage[utf8]{inputenc}
\usepackage[a4paper,margin=1in]{geometry}
\usepackage{lmodern}
\usepackage{microtype}
\usepackage{amsmath,amssymb}
\usepackage{booktabs}
\usepackage{graphicx}
\usepackage{hyperref}
\usepackage{natbib}

\title{A Minimal Architectural Proposal for Consciousness and Agency on a Shared Neural Backbone}
\author{Gunnar Zarncke (AE Studio)}
\date{August 2025}

\begin{document}
\maketitle

\begin{abstract}
We propose a compact systems-level architecture that separates \emph{shared machinery} from \emph{differential add-ons} to describe the overlap and differences of consciousness and agency. 
We propose that the shared neural backbone of both comprises (i) a learned generative model, (ii) attention/gain control, and (iii) a short temporal buffer. 
On top of this, two neural processing stacks branch: a consciousness stack (global broadcast and metacognition) and an agency stack (valuation register and efference cascade). 
\end{abstract}

\section{Motivation and scope}
\textbf{Consciousness} is used here in a minimalist, operational sense: contents becoming globally available to disparate subsystems and accessible for report \citep{dehaene2014consciousness, dehaene2001global}. 
\textbf{Agency} denotes the capacity to initiate and control actions according to goals, treating systems as if they have beliefs and desires when this makes good predictions for behavior \citep{dennett1987intentional, dennett1971intentional}. 
The two frequently co-occur in humans but neither logically entails the other; the architecture below explains the overlap and the dissociations.

The proposal is intentionally model-agnostic. It does not presuppose any specific theory of consciousness or control, and it requires only three building blocks that are broadly accepted in computational neuroscience: 
predictive modelling \citep{Friston2010}, attention-like gain control \citep{sarter2023attention}, and short-horizon working memory \citep{miller2015}.

\section{Backbone and add-ons: informal specification}
Let $s_t$ be latent states, $o_t$ sensations, and $a_t$ actions. 
The backbone maintains a predictive mapping $p(o_{t+1}\mid s_t)$ and a belief update $q(s_{t+1}\mid s_t,o_{t+1})$ over a \emph{buffer} of length $\tau\!\in\![0,\,2\,\mathrm{s}]$. 
A generic attention/gain process $\alpha_t$ modulates precision.

Two add-on stacks branch from this backbone:
\begin{itemize}
  \item \textbf{Consciousness stack}: (C1) \emph{Global broadcast hub}---a routing mechanism that makes selected contents available system-wide; (C2) \emph{Metacognitive monitor}---confidence/error-likelihood tagging of those contents.
  \item \textbf{Agency stack}: (A1) \emph{Valuation register}---a critic-like scalarising function over states and options \citep{salge2014empowerment}; (A2) \emph{Efference cascade}---an actor-like path from selected policies to motor/effector channels with forward models (efference copy) \citep{popa2018cerebellum}.
\end{itemize}

\section{Diagram: shared backbone with differential add-ons}
\begin{figure}[h]
\centering
\includegraphics[width=0.8\textwidth]{consciousness_agency_backbone.png}
\caption{Schematic separation of common machinery (backbone) from differential add-ons for consciousness and agency. The same predictive core serves both stacks; dissociations arise when an add-on fails or is absent.}
\label{fig:backbone}
\end{figure}

\section{Functional roles and minimal predictions}
Table~\ref{tab:roles} states each component’s role and gives testable signatures that do not depend on a specific neural implementation.

\begin{table}[ht]
\centering
\footnotesize
\begin{tabular}{@{}l p{5.2cm} p{6.2cm}@{}}
\toprule
Component & Functional role & Discriminating operational signatures \\
\midrule
Generative model & Maintain a compact, predictive scene state; bind multi-modal inputs over \,$\sim$\,2 s \cite{burgess2019monet,greff2019iodine,locatello2020slot} & Violations trigger transient error bursts; imagery substitutes for input with matched dynamics \\
Attention/gain & Allocate precision to task-relevant channels \cite{locatello2020slot} & Manipulating gain selectively boosts reportability or control without changing stimuli \\
Temporal buffer & Provide a rolling workspace for binding and comparison \cite{kirchhoff2018markov,Kirchhoff2018} & Lengthening/shortening window shifts integration of rapid sequences (e.g., speech, actions) \\
Global broadcast (C1) & Make selected contents globally available \cite{kirchhoff2018markov,Kirchhoff2018} & Loss yields competent behaviour without access/report ("islands"), or vice versa \\
Metacognition (C2) & Tag contents with confidence/error likelihood \cite{Kirchhoff2018} & Dissociable from task performance; manipulations shift confidence at fixed accuracy \\
Valuation register (A1) & Scalarise options via value or cost \cite{Kirchhoff2018} & Predicts discounting curves and effort allocation; lesions flatten gradients \\
Efference cascade (A2) & Translate selected policies into actions with forward models \cite{Kirchhoff2018} & Loss yields intact intent/report but failed execution; efference copy suppresses self-generated reafference \\
Phenomenal report & Externalisation of selected contents for report or communication \cite{Kirchhoff2018} & Report channels can be blocked while internal access remains (e.g., output bottlenecks); readouts include verbal report, button press, or eye-tracking \\
Motor execution & Physical realisation of selected policies via corticobasal and cerebellar loops \cite{Kirchhoff2018} & TMS/lesion yields intention–execution gaps; efference copy suppresses self-generated reafference, and failures cause sensory instability \\
\bottomrule
\end{tabular}
\caption{Roles and generic signatures of backbone and add-ons.}
\label{tab:roles}
\end{table}

\section{Why the architecture explains co-occurrence and dissociation}
Because both stacks share the same predictive core, they tend to co-occur in organisms with rich models. 
Yet each add-on can fail independently:
\begin{itemize}
  \item (C1) breakdown: intact skilled behaviour with impoverished access/report.
  \item (A2) breakdown: intact experience of intending with impaired execution.
  \item (C2) breakdown: normal performance but unreliable confidence control.
  \item (A1) breakdown: intact perception but apathetic or chaotic policy selection.
\end{itemize}
These double-dissociations are the architectural expectation, not anomalies.

\section{Outputs: Phenomenal report and motor control}
\subsection*{Phenomenal report}
We use ``report'' operationally in the sense of Dehaene \citep{dehaene2014consciousness}: any externally verifiable channel by which globally broadcast contents are
read out (speech, button presses, saccade targets). The report channel is \emph{downstream} of broadcast and
metacognition; it can therefore fail independently. Minimal predictions: (i) selective output interference
(e.g., speech block) diminishes report without changing internal access; (ii) confidence manipulations shift
report thresholds at matched accuracy. 
This operational approach allows detection of consciousness across species, particularly mammals, through neural patterns comparable to humans during reportable experiences.

\subsection*{Motor control}
Motor execution is the physical realisation of selected policies. A forward model (efference copy) predicts
reafferent input so that perception remains stable during movement. Minimal predictions: (i) perturbing motor
pathways leaves intention judgements intact while degrading execution; (ii) disrupting efference copy induces
reafferent confusion (e.g., tickle paradox breakdown).

The agency stack's applicability extends across species where observable intentional behavior can be detected \citep{dennett1987intentional}. 


\paragraph{Summary.} A small set of shared mechanisms plausibly underwrite both consciousness and agency. Two compact add-on stacks account for their frequent coupling and their separable failure modes. This provides a neutral scaffold for cumulative experiments and for comparison with more detailed theories.

\bibliographystyle{plainnat}
\bibliography{refs}

\end{document}
